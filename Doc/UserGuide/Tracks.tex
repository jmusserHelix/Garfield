The purpose of \texttt{Track...} classes is to simulate  
ionization patterns produced by charged particles traversing the detector. 

The type of the primary particle is set by the function
\begin{lstlisting}
void SetParticle(std::string particle);
\end{lstlisting}
\begin{description}
  \item[particle]
  name of the particle 
\end{description}
Only particles which are sufficiently long lived to leave a 
track in a detector are considered.
A list of the available particles is given 
in Table~\ref{Tab:ParticleNames}.

The kinematics of the charged particle can be defined 
by means of a number of equivalent methods:
  \begin{itemize}
    \item
    the total energy (in eV) of the particle,
    \item
    the kinetic energy (in eV) of the particle,
    \item
    the momentum (in eV/\(c\)) of the particle,
    \item
    the (dimension-less) velocity \(\beta = v / c\), 
    the Lorentz factor \(\gamma = 1 / \sqrt{1 - \beta^2}\) 
    or the product \(\beta\gamma\) of these two variables.
  \end{itemize}
The corresponding functions are
\begin{lstlisting}
void SetEnergy(const double e);
void SetKineticEnergy(const double ekin);
void SetMomentum(const double p);
void SetBeta(const double beta);
void SetGamma(const double gamma);
void SetBetaGamma(const double bg);
\end{lstlisting}

\begin{table}
  \centering
  \caption{Available charged particles.}
  \label{Tab:ParticleNames}
  \begin{tabular}{l l . r}
    \toprule
    particle & & \multicolumn{1}{c}{mass [MeV/\(c^{2}\)]} & charge \\
    \midrule
    \(e\)     & \texttt{electron, e-} & 0.510998910 & \(-1\) \\
    \(e^{+}\) & \texttt{positron, e+} & 0.510998910 & \(+1\) \\
    \(\mu^{-}\) & \texttt{muon, mu-}  & 105.658367  & \(-1\) \\
    \(\mu^{+}\) & \texttt{mu+}        & 105.658367  & \(+1\) \\
    \(\pi^{-}\) & \texttt{pion, pi, pi-} & 139.57018 & \(-1\) \\
    \(\pi^{+}\) & \texttt{pi+}           & 139.57018 & \(+1\) \\
    \(K^{-}\)   & \texttt{kaon, K, K-}   & 493.677   & \(-1\) \\
    \(K^{+}\)   & \texttt{K+}            & 493.677   & \(+1\) \\
    \(p\)       & \texttt{proton, p}     & 938.272013 & \(+1\) \\
    \(\overline{p}\) & \texttt{anti-proton, antiproton, p-bar} & 938.272013 & \(-1\) \\
    \(d\)       & \texttt{deuteron, d} & 1875.612793 & \(+1\) \\
  \bottomrule
  \end{tabular}
\end{table}

A track is initialized by means of
\begin{lstlisting}
void NewTrack(const double x0, const double y0, const double z0, const double t0,
              const double dx0, const double dy0, const double dz0);
\end{lstlisting}
\begin{description}
  \item[x0, y0, z0] initial position (in cm)
  \item[t0] starting time
  \item[dx0, dy0, dz0] initial direction vector
\end{description}
The starting point of the track has to be inside an ionizable medium. 
Depending on the type of \texttt{Track} class, there can be 
further restrictions on the type of \texttt{Medium}.
If the specified direction vector has zero length, an isotropic random vector 
will be generated.
 
After successful initialization, the ``clusters'' produced along the track
can be retrieved using
\begin{lstlisting}
const std::vector<Cluster>& GetClusters();
\end{lstlisting}
In this context, ``cluster'' refers to the energy loss in a single ionizing 
collision of the primary charged particle and the secondary 
electrons produced in this process. 
The concrete implementation of the cluster objects depends on the  
\texttt{Track} class.
  
\section{Heed}\label{Sec:Heed}

The program Heed \cite{Smirnov2005} is an implementation 
of the photo-absorption ionization (PAI) model. 
It was written by I. Smirnov.
An interface to Heed is available through the class \texttt{TrackHeed}. 

The \texttt{Cluster} objects returned by \texttt{TrackHeed::GetClusters} 
contain the position and time of the ionizing collision 
(member variables \texttt{x, y, z, t}), the transferred energy 
(member variable \texttt{energy}), and a \texttt{vector} of 
\texttt{Electron} objects corresponding to the conduction electrons 
associated to the cluster. 

In the following snippet, we iterate over the clusters along a track 
and the conduction electrons in each cluster.
\begin{lstlisting}
TrackHeed track;
// ...
double x0 = 0., y0 = 0., z0 = 0., t0 = 0.;
track.NewTrack(x0, y0, z0, t0, 0., 0., 0.);
// Loop over the clusters along the track.
for (const auto& cluster : track.GetClusters()) {
  // Loop over the conduction electrons in the cluster.
  for (const auto& electron : cluster.electrons) {
    // Get the coordinates of the electron.
    const double xe = electron.x; 
    const double ye = electron.y; 
    const double ze = electron.z; 
    const double te = electron.t; 
  }
}
\end{lstlisting}

\subsection{Delta electron transport}

Heed simulates the energy degradation of \(\delta\) electrons and 
the production of secondary (``conduction'') electrons 
using a phenomenological algorithm described in Ref.~\cite{Smirnov2005}.

\texttt{TrackHeed} retrieves the necessary input parameters - the 
asymptotic \(W\) value (eV) and the Fano factor - 
from the relevant \texttt{Medium} object. 
%\begin{lstlisting}
%void Medium::SetW(const double w);
%void Medium::SetFanoFactor(const double f);
%\end{lstlisting}
If these parameters are set to zero, Heed uses internal default values. 
The default value for the Fano factor is \(F = 0.19\).

The transport of \(\delta\) electrons can be activated or deactivated 
using
\begin{lstlisting}
void EnableDeltaElectronTransport();
void DisableDeltaElectronTransport();
\end{lstlisting} 

If \(\delta\) electron transport is disabled, 
the number of electrons returned by \texttt{GetCluster} is 
the number of ``primary'' ionisation electrons, 
\textit{i.\,e.}~the photo-electrons and Auger electrons. 
Their kinetic energies and locations are accessible 
through the function \texttt{GetElectron}.

If \(\delta\) electron transport is enabled (default setting), 
the function \texttt{GetElectron} returns the 
locations of the ``conduction'' electrons as calculated by the 
internal \(\delta\) transport algorithm of Heed. 
Since this method does not provide the energy and direction of the 
secondary electrons, the corresponding parameters in 
\texttt{GetElectron} are not meaningful in this case. 

\subsection{Photon transport}

Heed can also be used for simulating x-ray photoabsorption. 
\begin{lstlisting}
void TransportPhoton(const double x0, const double y0, const double z0,
                     const double t0, const double e0,
                     const double dx0, const double dy0, const double dz0,
                     int& nel);
\end{lstlisting}
\begin{description}
\item[x0, y0, z0, t0] initial position and time of the photon
\item[e0] photon energy in eV
\item[dx0, dy0, dz0] direction of the photon
\item[nel] number of photoelectrons and Auger-electrons produced 
           in the photon conversion
\end{description}

\subsection{Magnetic fields}
If the sensor has a non-zero magnetic field, \texttt{TrackHeed} will, 
by default, take the magnetic field into account for 
calculating the charged-particle trajectory.  
In order to explicitly enable or disable the use of the magnetic field  
in the stepping algorithm, the functions 
\begin{lstlisting}
EnableMagneticField();
DisableMagneticField();
\end{lstlisting}
can be called before simulating a track.
Depending on the strength of the magnetic field, it might be necessary to adapt the 
limits/parameters used in the stepping algorithm in order to obtain a smoothly curved 
trajectory as, for instance, in the example below.
\begin{lstlisting}
TrackHeed track;
// Get the default parameters.
double maxrange = 0., rforstraight = 0., stepstraight = 0., stepcurved = 0.;
track.GetSteppingLimits(maxrange, rforstraight, stepstraight, stepcurved);
// Reduce the step size [rad]. 
stepcurved = 0.02;
track.SetSteppingLimits(maxrange, rforstraight, stepstraight, stepcurved);
\end{lstlisting} 

\section{SRIM}
SRIM\footnote{Stopping and Range of Ions in Matter, \href{www.srim.org}{www.srim.org}} is a program for simulating the energy loss of ions in matter. 
It produces tables of stopping powers, range and straggling parameters that 
can subsequently be imported in Garfield using the class \texttt{TrackSrim}. 
The function
\begin{lstlisting}
bool ReadFile(const std::string& file)
\end{lstlisting}
returns \texttt{true} if the SRIM output file was read successfully.
The SRIM file contains the following data
\begin{itemize}
\item
a list of kinetic energies at which losses and straggling have been computed;
\item
average energy lost per unit distance via electromagnetic processes, for each energy;
\item
average energy lost per unit distance via nuclear processes, for each energy;
\item
projected path length, for each energy;
\item
longitudinal straggling, for each energy;
\item
transverse straggling, for each energy.
\end{itemize}
These can be visualized using the functions
\begin{lstlisting}
void PlotEnergyLoss();
void PlotRange();
void PlotStraggling();
\end{lstlisting}
and printed out using the function \texttt{TrackSrim::Print()}.
In addition to these tables, the file also contains the mass and charge of 
the projectile, and the density of the target medium.
These properties are also imported and stored by \texttt{TrackSrim} 
when reading in the file. Unlike in case of Heed, the particle type 
therefore does not need to be specified by the user. 
The user does however need to set the kinetic energy of the projectile.

\texttt{TrackSrim} tries to generate individual tracks which statistically 
reproduce the average quantities calculated by SRIM.
Starting with the energy specified by the user, it iteratively
\begin{itemize}
\item
computes (by interpolating in the tables)  
the electromagnetic and nuclear energy loss per unit length at the 
current particle energy,
\item
calculates a step with a length over which the particle will 
produce on average a certain number of electrons,
\item
updates the trajectory based on the longitudinal and transverse scatter 
at the current particle energy,
\item 
calculates a randomised actual electromagnetic energy loss over the step 
and updates the particle energy.
\end{itemize}
This is repeated until the particle has no energy left or leaves the geometry.
The model for randomising the energy loss over a step can be set using 
the function 
\begin{lstlisting}
void SetModel(const int m);
\end{lstlisting}
\begin{description}
\item[m] fluctuation model to be used (Table~\ref{Tab:SrimFluctuationModels}); the default setting is model 4.
\end{description}
\begin{table}
  \centering
  \caption{Fluctuation models in \texttt{TrackSrim}.}
  \label{Tab:SrimFluctuationModels}
  \begin{tabular}{l l}
    \toprule
    Model & Description \\
    \midrule
    0 & No fluctuations \\
    1 & Untruncated Landau distribution \\
    2 & Vavilov distribution (provided the kinematic parameters are within the range of applicability, \\
      & otherwise fluctuations are disabled) \\
    3 & Gaussian distribution \\
    4 & Combination of Landau, Vavilov and Gaussian models, \\
      & each applied in their alleged domain of applicability \\
    \bottomrule
  \end{tabular}
\end{table}
The generation of Vavilov distributed random numbers is based on a 
C++ implementation of the CERNLIB G115 procedures for the 
fast, approximate calculation of functions related to the 
Vavilov distribution. The description of the algorithm can be found 
in Ref.~\cite{RotondiMontagna1990}.

For sampling the energy loss, \texttt{TrackSrim} needs the 
electron density of the target material. By default it is retrieved 
from the relevant \texttt{Medium} object (and scaled to the mass density 
given in the SRIM output file). 
Alternatively, the user can specify the effective atomic number $Z$ 
and mass number $A$ of the target 
using \texttt{TrackSrim::SetAtomicMassNumbers}.


Transverse and longitudinal straggling can be switched on or off using
\begin{lstlisting}
void EnableTransverseStraggling(const bool on);
void EnableLongitudinalStraggling(const bool on);
\end{lstlisting}
If energy loss fluctuations are used, longitudinal straggling should be disabled.
By default, transverse straggling is switched on and longitudinal straggling 
is switched off. 

SRIM is aimed at low energy nuclear particles which deposit large numbers of electrons in a medium. 
The grouping of electrons to a cluster is therefore somewhat arbitrary. 
By default, \texttt{TrackSrim} will adjust the step size such that 
there are on average 100 clusters on the track.
If the user specifies a target cluster size, using
\begin{lstlisting}
void SetTargetClusterSize(const int n);
\end{lstlisting}
the step size will be chosen such that a cluster comprises on average 
\texttt{n} electrons. Alternatively, if the user specifies a maximum 
number of clusters, using
\begin{lstlisting}
void SetClustersMaximum(const int n);
\end{lstlisting}
the step size will be chosen such that on average there are 
\texttt{n / 2} clusters on the track.

To calculate the number of electrons for given amount of deposited energy,
\texttt{TrackSrim} needs the work function $W$ (in eV) and the 
Fano factor of the target material. 
By default, \texttt{TrackSrim} retrieves these parameters from the 
relevant \texttt{Medium} object.
Altenatively, they can be set explicitly using 
\texttt{TrackSrim::SetWorkFunction} and \texttt{TrackSrim::SetFanoFactor}.

\section{TRIM}
TRIM\footnote{TRansport of Ions in Matter} is a Monte Carlo simulation program 
from the same collection of software packages as SRIM. 
It simulates individual ion trajectories in a target 
(which can be made of several layers) and the processes following the ion's energy loss 
(recoil cascades, displacement damage, \textit{etc.}). 
TRIM typically produces a number of output files. 
One of them is a file called \texttt{EXYZ.txt} which lists the 
position and electronic stopping power for each simulated ion at 
regular steps in the ion's kinetic energy.
The energy interval is set by the user.

The class \texttt{TrackTrim} allows one to import these data in Garfield 
and use them for simulating tracks. The function for 
reading ion trajectories from an \texttt{EXYZ.txt} file is 
\begin{lstlisting}
bool ReadFile(const std::string& file, const unsigned int nIons = 0,
              const unsigned int nSkip = 0);
\end{lstlisting}
\begin{description}
  \item[file] name/path of the \texttt{EXYZ.txt} file to be loaded,
  \item[nIons] number of ion trajectories to be loaded from the file,
  \item[nSkip] number of ion trajectories to be skipped at the beginning of the file.
\end{description}
If the value of \texttt{nIons} is zero, \texttt{TrackTrim} will 
import all ion trajectories included in the file.

When the function \texttt{NewTrack} is called, 
\texttt{TrackTrim} will generate clusters according to one of the ion 
trajectories imported from \texttt{EXYZ.txt} (starting with the first one). 
Each cluster will correspond to one energy interval. 
The number of clusters along a track and the deposited energy per cluster 
is therefore controlled by the energy interval specified when running TRIM.
The number of electrons in a cluster is sampled using an algorithm 
that reproduces the requested work function ($W$ value) and Fano factor.
At the next call to \texttt{NewTrack}, \texttt{TrackTrim} moves to the  
next ion trajectory in the list (if it reaches the end of the list, 
it rewinds to the first one). 

\section{Degrade}
The class \texttt{TrackDegrade} simulates ionisation by primary electrons 
in a gas and the subsequent degradation of $\delta$ electrons, 
Auger electrons and photoelectrons, 
using an interface to the program \textsc{Degrade}, developed by S. Biagi. 
\textsc{Degrade} has many commonalities with \textsc{Magboltz}, 
in particular the database of electron-atom/molecule cross-sections. 

While the \textsc{Degrade} program can also be used for simulating X-rays, 
$\beta$ decay and double $\beta$ decay, but these features are not (yet) 
accessible through the \texttt{TrackDegrade} interface at the moment. 
It is also worth noting, that the current version of \texttt{TrackDegrade} 
does not consider the electric and magnetic fields in the detector.

The \texttt{Cluster} objects returned by \texttt{TrackDegrade::GetClusters} 
contain the position and time of the ionizing collision 
(member variables \texttt{x, y, z, t}), a \texttt{vector} of 
\texttt{Electron} objects corresponding to the thermalised electrons 
associated to the cluster. In addition, it also contains a \texttt{vector} 
of the $\delta$ electrons and Auger electrons.

\begin{lstlisting}
TrackDegrade track;
// ...
double x0 = 0., y0 = 0., z0 = 0., t0 = 0.;
double dx0 = 1., dy0 = 0., dz0 = 0.
track.NewTrack(x0, y0, z0, t0, dx0, dy0, dz0);
// Loop over the clusters along the track.
for (const auto& cluster : track.GetClusters()) {
  // Loop over the thermalised electrons in the cluster.
  for (const auto& electron : cluster.electrons) {
    // Get the coordinates and kinetic energy of the electron.
    double xe = electron.x;
    double ye = electron.y;
    double ze = electron.z;
    double te = electron.t;
    double ee = electron.energy;
  }
}
\end{lstlisting}

By default, electrons are tracked until their kinetic energy falls below 
2\,eV. This threshold can be changed using the function
\begin{lstlisting}
void SetThresholdEnergy(const double ethr);
\end{lstlisting}

If the function 
\begin{lstlisting}
void StoreExcitations(const bool on = true, const double ethr);
\end{lstlisting}
is called prior to \texttt{NewTrack}, the excitations (with excitation 
energy above \texttt{ethr}) produced by the primary and secondary electrons 
are also stored in the \texttt{Cluster} object. 
